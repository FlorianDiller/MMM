\documentclass{book}
\usepackage{fontspec}
\usepackage{graphicx}
\usepackage{amssymb}
\usepackage{wasysym}
\usepackage{enumitem}
\usepackage{fontawesome5}
\usepackage{xcolor}
\usepackage{comment}
\usepackage{tgbonum}
\usepackage{tikz}

\usepackage[margin=1.5cm]{geometry}

\newcommand{\playbookTitle}{}

\newcommand{\playbookImage}{}

\newcommand{\flavorText}{}

\newcommand{\charNames}{}

\newcommand{\charAnimals}{}

\newcommand{\charEnhancementOne}{}
\newcommand{\charEnhancementTwo}{}
\newcommand{\charEnhancementThree}{}
\newcommand{\charEnhancementFour}{}
\newcommand{\charEnhancementFive}{}
\newcommand{\charEnhancementSix}{}

\newcommand{\moveOne}{}
\newcommand{\moveTwo}{}
\newcommand{\moveThree}{}
\newcommand{\moveFour}{}
\newcommand{\moveFive}{}

\newcommand{\relationsOne}{}
\newcommand{\relationsTwo}{}
\newcommand{\relationsThree}{}

\newcommand{\leadingPrinciplesOne}{}
\newcommand{\leadingPrinciplesTwo}{}
\newcommand{\leadingPrinciplesThree}{}
\newcommand{\leadingPrinciplesFour}{}

\newcommand{\flipsideOne}{}
\newcommand{\flipsideTwo}{}
\newcommand{\flipsideThree}{}
\newcommand{\flipsideFour}{}

%\setmainfont{Arial}
\parindent0mm
\linespread{1.2}

\begin{document}
\pagenumbering{gobble}

\vspace*{\fill}

\begin{figure}[tph!]
\centering\includegraphics[width=12cm]{images/frontCover.png}
\end{figure}
\centering\Huge\fontspec{TradeWinds-Regular.ttf}Martial Mutant Misfitz!
\normalfont\large

A game by Florian Diller

\newpage

\pagenumbering{arabic}
\raggedright
\chapter*{\fontspec{TradeWinds-Regular.ttf}What is Martial Mutant Misfitz?}
\normalfont\large You are humanoid animal mutants. Human society rejects you. Your appearance is alien and off-putting to most humans. Fortunately, a mentor came into your life. Taking care of you when you needed it the most. Training you in martial prowess, and teaching you how to leverage your new abilities. A mentor hiding your from society and shielding you from harm. But this could soon end, as a new evil rises, that threatens the city. Who is gonna stop it? Who if not you?!\\
\medskip
Martial Mutant Misfitz is a game about teenagers mutated into whacky humanoid animals. It's a game about fighting with style and Kung Fu. It also is a game about being an outsider --- the feeling of not fitting into society. And lastly, it's a game about family and finding this feeling of belonging there.

\section*{\fontspec{TradeWinds-Regular.ttf}Motifs}
Science, Mutation, Ninjas, Samurai, Teenagers, Justice, Subculture, Family, Urban, Martial Arts


\chapter*{\fontspec{TradeWinds-Regular.ttf}Basic Moves}

\section*{\fontspec{TradeWinds-Regular.ttf}Kick Ass!}
When you fight a foe, roll +Grit.
\begin{itemize}
    \item \textbf{10+} -- You deal harm and avoid harm.
    \item \textbf{7-9} -- Trade harm.
\end{itemize}

\section*{\fontspec{TradeWinds-Regular.ttf}Defy Danger}
When you act despite an imminent threat, roll depending on the type of action:
\begin{itemize}
    \item Defending someone or something, Enduring or pushing through, +Grit
    \item Dodging, +Discipline
    \item Quick thinking, +Wits
    \item Doing a stunt, +Style
\end{itemize}
On a 10+, your action is successful. On a 7-9 you stumble, hesitate, or botch. The GM will offer you a worse outcome, a hard bargain, or ugly choice. 

\section*{\fontspec{TradeWinds-Regular.ttf}Talk the Talk}
When you persuade, bluff, or drop a catchphrase, roll +Wits.
\begin{itemize}
    \item \textbf{10+} -- They go along with it — at least for now.
    \item \textbf{7-9} -- They want something in return or remain suspicious.
\end{itemize}

\section*{\fontspec{TradeWinds-Regular.ttf}Radical Escape}
When you pull off a stunt or escape using vehicles, skates, or agility, roll +Style. On a 10+ choose one from below. On a 7-9 choose one from below, but you get a condition.
\begin{itemize}
    \item You impress someone.
    \item You get +1 forward.
    \item A teammate gets +1 forward.
\end{itemize}

\section*{\fontspec{TradeWinds-Regular.ttf}Team Up!}
When you assist a teammate, combine your powers, or do a signature team move, roll +Discipline. On a hit, choose one of the following according to the narrative:
\begin{itemize}
    \item You helped them, they get +1 forward.
    \item You both deal harm.
\end{itemize}
On a 7-9 you sacrifice something for the move: Positioning, Ressources or concentration. 

\section*{\fontspec{TradeWinds-Regular.ttf}Blend In}
When you try not to draw attention, roll + Style. On a 10+ you blend in. On a 7-9 choose one of the following:
\begin{itemize}
    \item Somebody gets suspicious.
    \item You draw a lot of attention.
    \item You stick out like a sore thumb.
    \item You make it but get a condition.
\end{itemize}

\section*{\fontspec{TradeWinds-Regular.ttf}Always a Little Smarter}
When you try to find out what's going on, roll +Wits. On a 10+ the GM answers you two, on a 7-9 one of the questions below.
\begin{itemize}
    \item What happened here?
    \item What sort of enemy are we dealing with here?
    \item What can it do?
    \item Where did it go?
    \item What was it up to?
    \item What is being concealed here?
\end{itemize}
On a miss, you reveal some information to your enemy. The GM might ask you questions.\\
It is important, that your investigations and their results are plausible and consistent with what's happening. The GM might ask you "How do you find that out?". If you don't have a good answer, choose another question instead.

\chapter*{\fontspec{TradeWinds-Regular.ttf}Hideout Moves}
\section*{\fontspec{TradeWinds-Regular.ttf}Training Montage}
When you train with your mentor, explain how and in what discipline roll +Discipline. On a 10+ choose two of the following. On a 7-9 choose one.
\begin{itemize}
    \item You impress your mentor
    \item Get +1 Team Mojo
    \item Get +1 Forward
\end{itemize}

\section*{\fontspec{TradeWinds-Regular.ttf}Huddle}
When you play out your leading principles while discusssing previous or upcoming adventures, add one to the Team Mojo.
%Lieber würfeln?

\section*{\fontspec{TradeWinds-Regular.ttf}Always a Little Smarter}
When you try to find out what's going on, roll +Wits. On a 10+ the GM answers you two, on a 7-9 one of the questions below.
\begin{itemize}
    \item How do we get in?
    \item How do we get out?
    \item What is a secret?
    \item In which area is the target?
    \item What danger we are likely to face?
\end{itemize}
On a miss, the target moves, information was wrong or an obstacles surfaces.

\section*{\fontspec{TradeWinds-Regular.ttf}Let's Roll!}
When you move out of the hideout to go on a mission, roll. Add +1 for everything that gives you an advantage for this mission -- information, gear, contacts, etc.
\begin{itemize}
    \item \textbf{10+} -- You are in a controlled situation when the action starts.
    \item \textbf{7-9} -- You are in a risky situation when the action starts.
    \item \textbf{6-} -- You are in a desperate situation when the action starts.
\end{itemize}

\chapter*{\fontspec{TradeWinds-Regular.ttf}Running the Game}
\section*{\fontspec{TradeWinds-Regular.ttf}Game Structure}
The structure of the game is seperated in to phases that correspond each to a different scenery.
\begin{itemize}
    \item Inside -- Hideout: Training, Planning, Discussing, Family, Mentor
    \item Outside -- The City: Fighting, Investigating, Action
\end{itemize}
In the hideout, the team interacts predominantly with one another and their mentor. Often, this represents a good opportunity to play out the \emph{Leading Principles} of the playbook. The mentor represents a useful narrative tool to you as a GM. Use him to steer the mutants in certain directions: Give them adventure hooks, send them out, be emotional, treat them like your children, allow them things, forbid them others.\\
\medskip
The session is usually best ended by the team having their favoutire junk food in their hideout. Don't forget the cliffhangers, tho! A cut to the evil guy does wonders.\\
\medskip
Like other games, Martial Mutant Misfitz can be a bit more fun, when you try to treat it like a series: Look for a theme song and play it in the beginning. Probably your game will take food, cigarette or toilet breaks. Try to transition back in with some fun advertisement. For example, narrate a 90s action figure commercial for the player character. Ask them details about the ad and the figure. Cereal commercials are great, too! Depending on your players you can also let them do the narration.

\section*{\fontspec{TradeWinds-Regular.ttf}GM Moves}
The GM has a seperate set of rules. They mostly come in play, when a player rolls a 6 or lower. There isn't necessarily a rule which of these moves has to be taken, except for "follow the narration".
\begin{itemize}
    \item Separate them.
    \item Shift the odds.
    \item Reveal future badness (something is building up).
    \item Reveal off-screen badness (sirens, smoke, TV, radio, …).
    \item Make them pay a price.
    \item Activate their stuffs' downside.
    \item Capture someone.
    \item Tell them the possible consequences and ask if they want to go ahead.
    \item Turn their move back on them.
    \item Offer an opportunity, maybe with a cost.
    \item Put someone in trouble.
    \item After every move, ask what they do next.
    \item Inflict harm or condition.
    \item Make them trade.
    \item Take away some of the mutants' stuff.
    \item Decrease the Team Mojo.
\end{itemize}

\section*{\fontspec{TradeWinds-Regular.ttf}Agenda}
There are certain rules, that everyone at the table agrees to follow. Some points to flavor the play, form the world and help make decisions. Additionally to the Agenda, every playbook has their Leading Principles, which should help the players roleplay their character.
\begin{itemize}
    \item Make the world seem real.
    \item Play to see what happens.
    \item Make the mutants' lives chaotic and rad.
    \item Portrait the life of teenage outsiders.
\end{itemize}

\section*{\fontspec{TradeWinds-Regular.ttf}Team Mojo}
The Team Mojo describes the coherence of the Team. Depending on the team coherence, assisting other team members, combining powers or team moves are becoming easier or harder. The Team Mojo ranges from -3 to 3. If the Team Mojo would be added while on 3, the player gets +1 forward. If the Team Mojo would be decreased under -3 a Team Rift occurs.

\section*{\fontspec{TradeWinds-Regular.ttf}Team Rift}
When the Team Mojo falls under -3, the team coherence is shattered. Nothing is coordinated and the team is being torn apart. This should have drastic consequences for the team in the narrative, such as being captured, seperated or worse.

\chapter*{\fontspec{TradeWinds-Regular.ttf}Gamemasters Section}
\section*{\fontspec{TradeWinds-Regular.ttf}Hideout vs. Action Dynamic}


\section*{\fontspec{TradeWinds-Regular.ttf}Setting the Scene}
To set the scene, try to invoke the feeling of a 90s action cartoon. That is being best done while describe the intro of your cartoon. When playing the first session or one shots, begin by playing a prepared rock song (Paul Gilbert is great for something like this) and start describing the logo of the cartoon and actions they take together. The intro then shows typical scenes in the lifes of each character. Go around the table and let each player introduce their character with a typical cartoon scene. When playing further sessions, you can let one, some, or all players describe some "last episode on"-bit.

This can be expanded on by describing to commercial breaks (normally in combination with cliffhangers in the session / episode). The GM should set an example by explaining the first commercial. The themes usually follow commercial in kids programs: Action figures, sweets, drinks, toys, etc. Afterwards, the GM can then ask if somebody has an idea for a commercial, prompt them to do a commercial, or ask what a commercial for an action figure of their character would look like.

\newpage
%begin{comment}
\renewcommand{\playbookTitle}{Aberrant}

\renewcommand{\playbookImage}{images/aberrant.png}

\renewcommand{\flavorText}{
\textit{Big muscles, bigger attitude. Always ready to rumble.}\\
\medskip
You mutated a bit too far. Your once beautiful face and body is now hideous. We're talking spikes, ooze, hunchback, you got the whole program. It's unthinkable that you go near human society. On the flipside, you received some crazy powers...\\
\medskip
You often need special attention and care. May it be the amount of food you consume, to find clothes in your size or that you crushed something by accident... again.
}

\renewcommand{\charNames}{1.Crux, 2.Maw, 3.Slitha, 4.Acida, 5.Scabarella, 6.Gnarl, \rule{2cm}{1pt}}

\renewcommand{\charAnimals}{1.Shark, 2.Worm, 3.Vulture, 4.Spider, 5.Crab, 6.Octopus, \rule{2cm}{1pt}}

\renewcommand{\charEnhancementOne}{\textbf{Razorsharp Spikes} (2 harm, hand, sharp)}
\renewcommand{\charEnhancementTwo}{\textbf{Extra limbs} (Once per scene, roll twice on \emph{Kick Ass!})}
\renewcommand{\charEnhancementThree}{\textbf{Back-Mounted Chemical Feed} (+1 Grit)}
\renewcommand{\charEnhancementFour}{\textbf{Rocklike Skin} (1 armor)}
\renewcommand{\charEnhancementFive}{\textbf{Hyper-Acid Digestion Track} (You can eat and digest everything.)}
\renewcommand{\charEnhancementSix}{\textbf{Amphibian Lung Gills}  (Can breath in and out of water, probably even more fluids.)}
%vertebrae whip 2 harm close

\renewcommand{\moveOne}{\textbf{It's not a glitch --- it's a feature.} Your mutation is unstable. Choose one of your enhancements and roll +Grit. On a 10+, exchange the chosen enhancement with another \playbookTitle enhancement. On a 7-9, roll for the new enhancement.}
\renewcommand{\moveTwo}{\textbf{This will just sting a little.} When you spend some time with a living or dead enemy, you can read their memories. Roll +Discipline. On a 7-9, ask the GM one of the following questions, on a 10+ two.
\vspace{-6pt}
\begin{itemize}
    \setlength\itemsep{-0.5em}
    \item Who do they work for?
    \item Where do they live?
    \item What is their favourite food?
    \item What was their childhood like?
    \item Do they have loved ones?
\end{itemize}}
\renewcommand{\moveThree}{\textbf{Walls are optional.} When you break through walls, destroy or throw obstacles. Roll +Grit. On a hit, choose one of the following:
\vspace{-6pt}
\begin{itemize}
    \setlength\itemsep{-0.5em}
\item Deal 2 harm
\item Deal 1 harm, area
\item Get or give +1 forward
\item You draw attention - enemies focus on you.
\item You create a advantageous position for your team.
\item You create a new "shortcut"
\end{itemize}
\vspace{-6pt}
On a 7-9 additionally choose one of the following, according to the narrative:
\vspace{-6pt}
\begin{itemize}
    \setlength\itemsep{-0.5em}
\item  Take 1 harm
\item  Take 1 condition
\item  Decrease the Team Mojo by 1
\end{itemize}

}
\renewcommand{\moveFour}{\textbf{Beauty's overrated. I went for memorable.} If a non hostile sees you for the first time, roll +Grit. On a hit choose one: They are stunned, afraid or fascinated. On a 7-9 they additionally react poorly.}
\renewcommand{\moveFive}{\textbf{Here, take some of my chemicals. They make you strong.} When you give some of your stabilizing chemicals to an ally, roll +Style. On a hit, they get +1 forward. On a 7-9 they additionally become poisoned.}

\renewcommand{\relationsOne}{\rule{2cm}{1pt} accidently used some of your stabilization chemicals. Ask the player how it came to this. Explain what temporary effect they had on them.}
\renewcommand{\relationsTwo}{You broke something dear of \rule{2cm}{1pt}. Ask the player what it was.}
\renewcommand{\relationsThree}{You and \rule{2cm}{1pt} share an absolutely weird hobby. What is it?}

\renewcommand{\leadingPrinciplesOne}{Need special treatment}
\renewcommand{\leadingPrinciplesTwo}{Your mutations behave weirdly}
\renewcommand{\leadingPrinciplesThree}{Mutate in weird ways}
\renewcommand{\leadingPrinciplesFour}{Don't fit}

\renewcommand{\flipsideOne}{Create problems}
\renewcommand{\flipsideTwo}{Draw attention}
\renewcommand{\flipsideThree}{Break something}
\renewcommand{\flipsideFour}{Feel left alone}

\pagenumbering{gobble}

\vspace*{\fill}

\begin{figure}[h!]
\centering\includegraphics[width=12cm]{\playbookImage}
\vspace{-\baselineskip}\vspace{+0.1pt}
\rule{\linewidth}{2pt}
\end{figure}
\Huge\fontspec{TradeWinds-Regular.ttf}The \playbookTitle

\normalfont\large
\medskip

\flavorText

\newpage

\Large\fontspec{TradeWinds-Regular.ttf}Animal

\medskip

\normalfont\large \charAnimals

\medskip

\Large\fontspec{TradeWinds-Regular.ttf}Names

\medskip

\normalfont\large \charNames

\medskip

\Large\fontspec{TradeWinds-Regular.ttf}Enhancements

\medskip

\normalfont\large Pick or roll two:

\(\square\) \charEnhancementOne

\(\square\) \charEnhancementTwo

\(\square\) \charEnhancementThree

\(\square\) \charEnhancementFour

\(\square\) \charEnhancementFive

\(\square\) \charEnhancementSix

\medskip

\Large\fontspec{TradeWinds-Regular.ttf}Advancements \(\LARGE \Circle \Circle \Circle \Circle \Circle \)

\medskip

\normalfont\large

\begin{tabular}{l @{\hspace{2cm}} l}
\(\square\) Get +1 Power, max +3 & \(\square\) Take another \playbookTitle~move \\
\(\square\) Get +1 Cool, max +3 & \(\square\) Take another \playbookTitle~enhancement \\
\(\square\) Get +1 Wits, max +3 & \(\square\) Take a move from another playbook \\
\(\square\) Get +1 Heart, max +3 & \(\square\) Take a move from another playbook \\
\(\square\) Get +1 Weird, max +3 & \\
\end{tabular}

\medskip

\Large\fontspec{TradeWinds-Regular.ttf}Conditions

\medskip

\normalfont\large

\(\square\) \textbf{Exposed} (-2 to Power until you eliminate or evade the skeptical)\\
\(\square\) \textbf{Angry} (-2 to Cool until you hurt someone or break something)\\
\(\square\) \textbf{Stressed} (-2 to Weird until you say sth hurtful to someone)\\
\(\square\) \textbf{Jealous} (-2 to Heart until you go on an ego trip)\\
\(\square\) \textbf{Insecure} (-2 to Charm until you take a comment to wrong way)\\
\(\square\) \textbf{Poisoned}  (-1 to all stats until healed)

\medskip

\Large\fontspec{TradeWinds-Regular.ttf}Harm \(\bigtriangledown \bigtriangledown \bigtriangledown \bigtriangledown | \bigtriangledown \bigtriangledown\)

\medskip

\Large\fontspec{TradeWinds-Regular.ttf}Stats

\normalfont\Huge

%Add tabular?
\faBomb~\textcolor{lightgray}{\faCircle[regular]} \hspace{0.5cm} \faStar~\textcolor{lightgray}{\faCircle[regular]} \hspace{0.5cm} \faHeart~\textcolor{lightgray}{\faCircle[regular]} \hspace{0.5cm} \faBrain~\textcolor{lightgray}{\faCircle[regular]} \hspace{0.5cm} \faPizzaSlice~\textcolor{lightgray}{\faCircle[regular]}

\medskip

\Large\fontspec{TradeWinds-Regular.ttf}Notes

\newpage

\Large\fontspec{TradeWinds-Regular.ttf}\playbookTitle~Moves

\medskip

\normalfont\large

\begin{itemize}[label=$\square$]

\item \moveOne

\item \moveTwo

\item \moveThree

\item \moveFour

\item \moveFive

\end{itemize}


\Large\fontspec{TradeWinds-Regular.ttf}Relations

\medskip

\normalfont\large

\begin{itemize}[label=$\square$]
    \item \relationsOne
    \item \relationsTwo
    \item \relationsThree
\end{itemize}

\begin{tabular}{l @{\hspace{2cm}} l}

\Large\fontspec{TradeWinds-Regular.ttf}Leading Principles & \Large\fontspec{TradeWinds-Regular.ttf}Flipside \medskip\\

\normalfont\large

$\bullet$ \leadingPrinciplesOne & $\bullet$ \flipsideOne \\
$\bullet$ \leadingPrinciplesTwo &  $\bullet$ \flipsideTwo \\
$\bullet$ \leadingPrinciplesThree &  $\bullet$ \flipsideThree \\
$\bullet$ \leadingPrinciplesFour &  $\bullet$ \flipsideFour \\

\end{tabular}

\newpage
\renewcommand{\playbookTitle}{Brainiac}

\renewcommand{\playbookImage}{images/brainiac.png}

\renewcommand{\flavorText}{
\textit{"For every problem there is a solution and I will find it."}\\
\medskip
You love science and solving puzzles. You are the brains of your team.\\
\medskip
Towards the others you often pled for waiting to receive more data, diskussing and analyzing.
}

\renewcommand{\charNames}{1.Cypher, 2.Echo, 3.Vector, 4.Ivy, 5.Pixel, 6.Glitch, \rule{2cm}{1pt}}

\renewcommand{\charAnimals}{1.Octopus, 2.Raven, 3.Owl, 4.Dolphin, 5.Fox, 6.Gorilla, \rule{2cm}{1pt}}

\renewcommand{\charEnhancementOne}{\textbf{MutaTech Mark IV Attack Drones} (1 harm, far, electric)}
\renewcommand{\charEnhancementTwo}{\textbf{Backpack-Lab} Gives you the opportunity to analyze and synthesize chemicals on the go.}
\renewcommand{\charEnhancementThree}{\textbf{MorphSwarm Units} (1 harm, hand, choose blunt, pierce, slash)}
\renewcommand{\charEnhancementFour}{\textbf{Hyper-Cognition} (choose one additional move)}
\renewcommand{\charEnhancementFive}{\textbf{Ultra-Adaptive NeuroShield} (1 armor)}
\renewcommand{\charEnhancementSix}{\textbf{EMP} (1 harm vs. mechanical, area)}

\renewcommand{\moveOne}{\textbf{This won’t take long.} Hack, repair or manipulate something. Roll +Discipline. On a hit you surpass the challenge. On a 7-9 you only get a limited time in the system.}
\renewcommand{\moveTwo}{\textbf{You can’t hide from thermals.} Roll +Wits. On a 10+, ask the GM two of the following questions, on a 7-9 one.
\vspace{-6pt}
\begin{itemize}
    \item Is there someone behind this wall?
    \item What would someone see if the visibility was better?
    \item How many individuals are we dealing with?
    \item What is a weak spot here?
    \item How do we get in or out?
    \item What is the safest way forward?
    \item What is some valuable piece of information?
\end{itemize}
}
\renewcommand{\moveThree}{\textbf{Hope they backed this up.} When you receive information or hack into a system, you get or give +1 forward.}
\renewcommand{\moveFour}{\textbf{Crafted with love... and a little bit of panic.} Roll +Wits. On a hit you improvise a weapon (1 harm) and get +1 forward. On a 7-9 it only lasts for this combat. Explain what it is made of and how it looks like.}
\renewcommand{\moveFive}{\textbf{Keep calm and pretend this is fine.} Roll +Discipline. On a hit you clear a condition off of you}

\renewcommand{\relationsOne}{\rule{2cm}{1pt} and you love the same video game. How is it called? What is it about?}
\renewcommand{\relationsTwo}{You taught \rule{2cm}{1pt} about the science field you love. What is it?}
\renewcommand{\relationsThree}{You found unusual data or browser history from \rule{2cm}{1pt} on the shared computer. Ask the player what it was about.}

\renewcommand{\leadingPrinciplesOne}{Be the rationale}
\renewcommand{\leadingPrinciplesTwo}{Solve puzzles}
\renewcommand{\leadingPrinciplesThree}{Provide information}
\renewcommand{\leadingPrinciplesFour}{Correct wrong or imprecise information}

\renewcommand{\flipsideOne}{Think too long}
\renewcommand{\flipsideTwo}{Don't act}
\renewcommand{\flipsideThree}{Be hesitant}
\renewcommand{\flipsideFour}{Have troubles deciding}

\pagenumbering{gobble}

\vspace*{\fill}

\begin{figure}[h!]
\centering\includegraphics[width=12cm]{\playbookImage}
\vspace{-\baselineskip}\vspace{+0.1pt}
\rule{\linewidth}{2pt}
\end{figure}
\Huge\fontspec{TradeWinds-Regular.ttf}The \playbookTitle

\normalfont\large
\medskip

\flavorText

\newpage

\Large\fontspec{TradeWinds-Regular.ttf}Animal

\medskip

\normalfont\large \charAnimals

\medskip

\Large\fontspec{TradeWinds-Regular.ttf}Names

\medskip

\normalfont\large \charNames

\medskip

\Large\fontspec{TradeWinds-Regular.ttf}Enhancements

\medskip

\normalfont\large Pick or roll two:

\(\square\) \charEnhancementOne

\(\square\) \charEnhancementTwo

\(\square\) \charEnhancementThree

\(\square\) \charEnhancementFour

\(\square\) \charEnhancementFive

\(\square\) \charEnhancementSix

\medskip

\Large\fontspec{TradeWinds-Regular.ttf}Advancements \(\LARGE \Circle \Circle \Circle \Circle \Circle \)

\medskip

\normalfont\large

\begin{tabular}{l @{\hspace{2cm}} l}
\(\square\) Get +1 Power, max +3 & \(\square\) Take another \playbookTitle~move \\
\(\square\) Get +1 Cool, max +3 & \(\square\) Take another \playbookTitle~enhancement \\
\(\square\) Get +1 Wits, max +3 & \(\square\) Take a move from another playbook \\
\(\square\) Get +1 Heart, max +3 & \(\square\) Take a move from another playbook \\
\(\square\) Get +1 Weird, max +3 & \\
\end{tabular}

\medskip

\Large\fontspec{TradeWinds-Regular.ttf}Conditions

\medskip

\normalfont\large

\(\square\) \textbf{Exposed} (-2 to Power until you eliminate or evade the skeptical)\\
\(\square\) \textbf{Angry} (-2 to Cool until you hurt someone or break something)\\
\(\square\) \textbf{Stressed} (-2 to Weird until you say sth hurtful to someone)\\
\(\square\) \textbf{Jealous} (-2 to Heart until you go on an ego trip)\\
\(\square\) \textbf{Insecure} (-2 to Charm until you take a comment to wrong way)\\
\(\square\) \textbf{Poisoned}  (-1 to all stats until healed)

\medskip

\Large\fontspec{TradeWinds-Regular.ttf}Harm \(\bigtriangledown \bigtriangledown \bigtriangledown \bigtriangledown | \bigtriangledown \bigtriangledown\)

\medskip

\Large\fontspec{TradeWinds-Regular.ttf}Stats

\normalfont\Huge

%Add tabular?
\faBomb~\textcolor{lightgray}{\faCircle[regular]} \hspace{0.5cm} \faStar~\textcolor{lightgray}{\faCircle[regular]} \hspace{0.5cm} \faHeart~\textcolor{lightgray}{\faCircle[regular]} \hspace{0.5cm} \faBrain~\textcolor{lightgray}{\faCircle[regular]} \hspace{0.5cm} \faPizzaSlice~\textcolor{lightgray}{\faCircle[regular]}

\medskip

\Large\fontspec{TradeWinds-Regular.ttf}Notes

\newpage

\Large\fontspec{TradeWinds-Regular.ttf}\playbookTitle~Moves

\medskip

\normalfont\large

\begin{itemize}[label=$\square$]

\item \moveOne

\item \moveTwo

\item \moveThree

\item \moveFour

\item \moveFive

\end{itemize}


\Large\fontspec{TradeWinds-Regular.ttf}Relations

\medskip

\normalfont\large

\begin{itemize}[label=$\square$]
    \item \relationsOne
    \item \relationsTwo
    \item \relationsThree
\end{itemize}

\begin{tabular}{l @{\hspace{2cm}} l}

\Large\fontspec{TradeWinds-Regular.ttf}Leading Principles & \Large\fontspec{TradeWinds-Regular.ttf}Flipside \medskip\\

\normalfont\large

$\bullet$ \leadingPrinciplesOne & $\bullet$ \flipsideOne \\
$\bullet$ \leadingPrinciplesTwo &  $\bullet$ \flipsideTwo \\
$\bullet$ \leadingPrinciplesThree &  $\bullet$ \flipsideThree \\
$\bullet$ \leadingPrinciplesFour &  $\bullet$ \flipsideFour \\

\end{tabular}

\newpage
\renewcommand{\playbookTitle}{Brawler}

\renewcommand{\playbookImage}{images/brawler.png}

\renewcommand{\flavorText}{
\textit{Big muscles, bigger attitude. Always ready to rumble.}\\
\medskip
You love fighting and your mutations gave you the tools to really excel at it. This gives you the power to protect others. However, your hot headedness brings you a lot of trouble.\\
\medskip
Towards the others you always argue for acting. Too much diskussion gives you headaches.
}

\renewcommand{\charNames}{1.Knuckles, 2.Jax, 3.Brock, 4.Roxy, 5.Bruiza, 6.Wrecka, \rule{2cm}{1pt}}

\renewcommand{\charAnimals}{1.Bear, 2.Rhino, 3.Buffalo, 4.Crocodile, 5.Kangaroo, 6.Pangolin, \rule{2cm}{1pt}}

\renewcommand{\charEnhancementOne}{\textbf{Electrified Brass Knuckles} (2 harm, 1 harm ignores armor electric, 1 harm blunt, hand)}
\renewcommand{\charEnhancementTwo}{\textbf{Spiked Wrist Wraps} (1 harm, quick, hand, blunt, pierce)}
\renewcommand{\charEnhancementThree}{\textbf{Tech-Spine} (makes all weapons quick)}
\renewcommand{\charEnhancementFour}{\textbf{Hardened Skin} (2 armor vs. piercing)}
\renewcommand{\charEnhancementFive}{\textbf{Razor Claws} (2 harm, hand, slash, pierce)}
\renewcommand{\charEnhancementSix}{\textbf{Steel Bones}  (2 armor vs. blunt)}

\renewcommand{\moveOne}{\textbf{One more word and I break your nose} If you try to provoke or intimidate someone you can roll +Grit on Talk the Talk}
\renewcommand{\moveTwo}{\textbf{Nothin' but a scratch} Once per session you can roll +Discipline. On a 10+ heal 2 harm. On a 7-9 heal 1 harm. On a miss, it was worse than it looked.}
\renewcommand{\moveThree}{\textbf{Walls are optional.} You are a master of fighting styles. Karate, Kung Fu, Ninjitsu, Tigerclaw, Mantis, Drunken Master, you name it. You can combine them and switch fluidly between them. Explain what you do and roll +Grit. On a 10+, choose two of the following:
\vspace{-6pt}
\begin{itemize}
    \setlength\itemsep{-0.5em}
    \item Deal 1 harm
    \item Get or give +1 forward
    \item You draw attention - enemies focus on you.
\end{itemize}
\vspace{-6pt}
On a 7-9 you choose one and additionally get one harm or become fractured.}
\renewcommand{\moveFour}{\textbf{Coming through!} When you charge at the enemy, you shrug off harm until your momentum stops.}
\renewcommand{\moveFive}{\textbf{You brought a gang? Cute.} Once per session, you can add area to any attack.}

\renewcommand{\relationsOne}{\rule{2cm}{1pt}'s and your fighting style derived from the same base style. How is it called? What does it look like? What is it about?}
\renewcommand{\relationsTwo}{You often get into trouble with \rule{2cm}{1pt}. Doing what?}
\renewcommand{\relationsThree}{When you were younger, \rule{2cm}{1pt} and you wrestled all the time. What famous wrestler did you embody?}

\renewcommand{\leadingPrinciplesOne}{Protect others}
\renewcommand{\leadingPrinciplesTwo}{Take action}
\renewcommand{\leadingPrinciplesThree}{Go solo}
\renewcommand{\leadingPrinciplesFour}{Take an ego trip}

\renewcommand{\flipsideOne}{Risk too much}
\renewcommand{\flipsideTwo}{Be impatient}
\renewcommand{\flipsideThree}{Freak out way too soon}
\renewcommand{\flipsideFour}{Think too little about a plan or problem}

\pagenumbering{gobble}

\vspace*{\fill}

\begin{figure}[h!]
\centering\includegraphics[width=12cm]{\playbookImage}
\vspace{-\baselineskip}\vspace{+0.1pt}
\rule{\linewidth}{2pt}
\end{figure}
\Huge\fontspec{TradeWinds-Regular.ttf}The \playbookTitle

\normalfont\large
\medskip

\flavorText

\newpage

\Large\fontspec{TradeWinds-Regular.ttf}Animal

\medskip

\normalfont\large \charAnimals

\medskip

\Large\fontspec{TradeWinds-Regular.ttf}Names

\medskip

\normalfont\large \charNames

\medskip

\Large\fontspec{TradeWinds-Regular.ttf}Enhancements

\medskip

\normalfont\large Pick or roll two:

\(\square\) \charEnhancementOne

\(\square\) \charEnhancementTwo

\(\square\) \charEnhancementThree

\(\square\) \charEnhancementFour

\(\square\) \charEnhancementFive

\(\square\) \charEnhancementSix

\medskip

\Large\fontspec{TradeWinds-Regular.ttf}Advancements \(\LARGE \Circle \Circle \Circle \Circle \Circle \)

\medskip

\normalfont\large

\begin{tabular}{l @{\hspace{2cm}} l}
\(\square\) Get +1 Power, max +3 & \(\square\) Take another \playbookTitle~move \\
\(\square\) Get +1 Cool, max +3 & \(\square\) Take another \playbookTitle~enhancement \\
\(\square\) Get +1 Wits, max +3 & \(\square\) Take a move from another playbook \\
\(\square\) Get +1 Heart, max +3 & \(\square\) Take a move from another playbook \\
\(\square\) Get +1 Weird, max +3 & \\
\end{tabular}

\medskip

\Large\fontspec{TradeWinds-Regular.ttf}Conditions

\medskip

\normalfont\large

\(\square\) \textbf{Exposed} (-2 to Power until you eliminate or evade the skeptical)\\
\(\square\) \textbf{Angry} (-2 to Cool until you hurt someone or break something)\\
\(\square\) \textbf{Stressed} (-2 to Weird until you say sth hurtful to someone)\\
\(\square\) \textbf{Jealous} (-2 to Heart until you go on an ego trip)\\
\(\square\) \textbf{Insecure} (-2 to Charm until you take a comment to wrong way)\\
\(\square\) \textbf{Poisoned}  (-1 to all stats until healed)

\medskip

\Large\fontspec{TradeWinds-Regular.ttf}Harm \(\bigtriangledown \bigtriangledown \bigtriangledown \bigtriangledown | \bigtriangledown \bigtriangledown\)

\medskip

\Large\fontspec{TradeWinds-Regular.ttf}Stats

\normalfont\Huge

%Add tabular?
\faBomb~\textcolor{lightgray}{\faCircle[regular]} \hspace{0.5cm} \faStar~\textcolor{lightgray}{\faCircle[regular]} \hspace{0.5cm} \faHeart~\textcolor{lightgray}{\faCircle[regular]} \hspace{0.5cm} \faBrain~\textcolor{lightgray}{\faCircle[regular]} \hspace{0.5cm} \faPizzaSlice~\textcolor{lightgray}{\faCircle[regular]}

\medskip

\Large\fontspec{TradeWinds-Regular.ttf}Notes

\newpage

\Large\fontspec{TradeWinds-Regular.ttf}\playbookTitle~Moves

\medskip

\normalfont\large

\begin{itemize}[label=$\square$]

\item \moveOne

\item \moveTwo

\item \moveThree

\item \moveFour

\item \moveFive

\end{itemize}


\Large\fontspec{TradeWinds-Regular.ttf}Relations

\medskip

\normalfont\large

\begin{itemize}[label=$\square$]
    \item \relationsOne
    \item \relationsTwo
    \item \relationsThree
\end{itemize}

\begin{tabular}{l @{\hspace{2cm}} l}

\Large\fontspec{TradeWinds-Regular.ttf}Leading Principles & \Large\fontspec{TradeWinds-Regular.ttf}Flipside \medskip\\

\normalfont\large

$\bullet$ \leadingPrinciplesOne & $\bullet$ \flipsideOne \\
$\bullet$ \leadingPrinciplesTwo &  $\bullet$ \flipsideTwo \\
$\bullet$ \leadingPrinciplesThree &  $\bullet$ \flipsideThree \\
$\bullet$ \leadingPrinciplesFour &  $\bullet$ \flipsideFour \\

\end{tabular}

\newpage
\renewcommand{\playbookTitle}{Human}

\renewcommand{\playbookImage}{images/human.png}

\renewcommand{\flavorText}{
\textit{I mean, yeah... they are a bit extraordinary, but they got character!}\\
\medskip
Good job! You're the odd one in a team of mutants. Your are the human. Your resources and knowledge concerning the human society are invaluable to the mutants.\\
\medskip
You supply the others with what they need. May it be their favourite junk food, the newest video games or some clothes to blend in, you can get it.
}

\renewcommand{\charNames}{1.Jessica, 2.Brian, 3.Emily, 4.Mike, 5.Amanda, 6.Dan, \rule{2cm}{1pt}}

\renewcommand{\charAnimals}{Human.}

\renewcommand{\charEnhancementOne}{\textbf{Ice Hockey Stick} (1 harm, hand, blunt)}
\renewcommand{\charEnhancementTwo}{\textbf{Slingshot} (1 harm, far, blunt)}
\renewcommand{\charEnhancementThree}{\textbf{Fast Food Uniform} (Unending supply of a junk food you determine)}
\renewcommand{\charEnhancementFour}{\textbf{School Books} (+1 Wits)}
\renewcommand{\charEnhancementFive}{\textbf{Leather Jacket} (1 armor)}
\renewcommand{\charEnhancementSix}{\textbf{Inherited Army Knife} (Once per session declare "Of course it can do that!")}

\renewcommand{\moveOne}{\textbf{I'm just unexcitingly normal.} You automatically \emph{Blend In} normal society with a 10+. If it's a place you shouldn't be, roll like normal.}
\renewcommand{\moveTwo}{\textbf{I may not have claws, but I can get a pizza delivered anywhere.} When you trying to supply someone with something roll +Style. On a hit, supply one of the following:
\vspace{-6pt}
\begin{itemize}
    \setlength\itemsep{-0.5em}
    \item An address
    \item Something you can buy normally
    \item Access to a public building
    \item Information about human society
    \item Means of transport
\end{itemize}
\vspace{-6pt}
On a 7-9 somebody gets suspicious about the way you obtain the thing or the amount, size, etc.
}
\renewcommand{\moveThree}{\textbf{Underdog? Nah, I'm the plot twist.} When you go against somebody clearly stronger or weirder, roll +Grit. On a 10+ your guts inspire an ally. You and that ally get +1 forward. On a 7-9 you stand your ground, but get a condition or harm.}
\renewcommand{\moveFour}{\textbf{Didn't bring it. Didn't need it. Probably...} When you go are completely unprepared, you can make do with what you got. Explain what exactly you do and roll +Wits. On a 10+ you deal 2 harm. On a 7-9 you deal 1 harm and get a condition.}
\renewcommand{\moveFive}{\textbf{Outplayed. Outclassed. Out of your league.} You make up what you lack in strength with Wits. When you \emph{Kick Ass}, you can roll +Wits when you explain how you outsmart them.}
%Ghetto Blaster move?


\renewcommand{\relationsOne}{\rule{2cm}{1pt} always tells you to get him something from the human world. Ask the player what.}
\renewcommand{\relationsTwo}{\rule{2cm}{1pt} brought you into the team. Explain how that happened.}
\renewcommand{\relationsThree}{You smuggled \rule{2cm}{1pt} into a place where they should not have been. Where? What were you doing?}

\renewcommand{\leadingPrinciplesOne}{Supply others}
\renewcommand{\leadingPrinciplesTwo}{Provide information about humans}
\renewcommand{\leadingPrinciplesThree}{Give opportunities}
\renewcommand{\leadingPrinciplesFour}{Know someone}

\renewcommand{\flipsideOne}{Be a know-it-all}
\renewcommand{\flipsideTwo}{Run out of something}
\renewcommand{\flipsideThree}{Shatter their expectations}
\renewcommand{\flipsideFour}{Something costs more than they thought}

\pagenumbering{gobble}

\vspace*{\fill}

\begin{figure}[h!]
\centering\includegraphics[width=12cm]{\playbookImage}
\vspace{-\baselineskip}\vspace{+0.1pt}
\rule{\linewidth}{2pt}
\end{figure}
\Huge\fontspec{TradeWinds-Regular.ttf}The \playbookTitle

\normalfont\large
\medskip

\flavorText

\newpage

\Large\fontspec{TradeWinds-Regular.ttf}Animal

\medskip

\normalfont\large \charAnimals

\medskip

\Large\fontspec{TradeWinds-Regular.ttf}Names

\medskip

\normalfont\large \charNames

\medskip

\Large\fontspec{TradeWinds-Regular.ttf}Enhancements

\medskip

\normalfont\large Pick or roll two:

\(\square\) \charEnhancementOne

\(\square\) \charEnhancementTwo

\(\square\) \charEnhancementThree

\(\square\) \charEnhancementFour

\(\square\) \charEnhancementFive

\(\square\) \charEnhancementSix

\medskip

\Large\fontspec{TradeWinds-Regular.ttf}Advancements \(\LARGE \Circle \Circle \Circle \Circle \Circle \)

\medskip

\normalfont\large

\begin{tabular}{l @{\hspace{2cm}} l}
\(\square\) Get +1 Power, max +3 & \(\square\) Take another \playbookTitle~move \\
\(\square\) Get +1 Cool, max +3 & \(\square\) Take another \playbookTitle~enhancement \\
\(\square\) Get +1 Wits, max +3 & \(\square\) Take a move from another playbook \\
\(\square\) Get +1 Heart, max +3 & \(\square\) Take a move from another playbook \\
\(\square\) Get +1 Weird, max +3 & \\
\end{tabular}

\medskip

\Large\fontspec{TradeWinds-Regular.ttf}Conditions

\medskip

\normalfont\large

\(\square\) \textbf{Exposed} (-2 to Power until you eliminate or evade the skeptical)\\
\(\square\) \textbf{Angry} (-2 to Cool until you hurt someone or break something)\\
\(\square\) \textbf{Stressed} (-2 to Weird until you say sth hurtful to someone)\\
\(\square\) \textbf{Jealous} (-2 to Heart until you go on an ego trip)\\
\(\square\) \textbf{Insecure} (-2 to Charm until you take a comment to wrong way)\\
\(\square\) \textbf{Poisoned}  (-1 to all stats until healed)

\medskip

\Large\fontspec{TradeWinds-Regular.ttf}Harm \(\bigtriangledown \bigtriangledown \bigtriangledown \bigtriangledown | \bigtriangledown \bigtriangledown\)

\medskip

\Large\fontspec{TradeWinds-Regular.ttf}Stats

\normalfont\Huge

%Add tabular?
\faBomb~\textcolor{lightgray}{\faCircle[regular]} \hspace{0.5cm} \faStar~\textcolor{lightgray}{\faCircle[regular]} \hspace{0.5cm} \faHeart~\textcolor{lightgray}{\faCircle[regular]} \hspace{0.5cm} \faBrain~\textcolor{lightgray}{\faCircle[regular]} \hspace{0.5cm} \faPizzaSlice~\textcolor{lightgray}{\faCircle[regular]}

\medskip

\Large\fontspec{TradeWinds-Regular.ttf}Notes

\newpage

\Large\fontspec{TradeWinds-Regular.ttf}\playbookTitle~Moves

\medskip

\normalfont\large

\begin{itemize}[label=$\square$]

\item \moveOne

\item \moveTwo

\item \moveThree

\item \moveFour

\item \moveFive

\end{itemize}


\Large\fontspec{TradeWinds-Regular.ttf}Relations

\medskip

\normalfont\large

\begin{itemize}[label=$\square$]
    \item \relationsOne
    \item \relationsTwo
    \item \relationsThree
\end{itemize}

\begin{tabular}{l @{\hspace{2cm}} l}

\Large\fontspec{TradeWinds-Regular.ttf}Leading Principles & \Large\fontspec{TradeWinds-Regular.ttf}Flipside \medskip\\

\normalfont\large

$\bullet$ \leadingPrinciplesOne & $\bullet$ \flipsideOne \\
$\bullet$ \leadingPrinciplesTwo &  $\bullet$ \flipsideTwo \\
$\bullet$ \leadingPrinciplesThree &  $\bullet$ \flipsideThree \\
$\bullet$ \leadingPrinciplesFour &  $\bullet$ \flipsideFour \\

\end{tabular}

\newpage
\renewcommand{\playbookTitle}{Leader}

\renewcommand{\playbookImage}{images/leader.png}

\renewcommand{\flavorText}{
\textit{The team is the most important. To me, you are my family.}\\
\medskip
You coordinate and bring the team together. You try to hear every voice in the team and try to compromise. In the field you often try to coordinate this chaotic crew of mutants.
}

\renewcommand{\charNames}{1.Rex, 2.Scar, 3.Cinder, 4.Ash, 5.Vax, 6.Crash, \rule{2cm}{1pt}}

\renewcommand{\charAnimals}{1.Turtle, 2.Rhino, 3.Bear, 4.Lion, 5.Falcon, 6.Mustang \rule{2cm}{1pt}}

\renewcommand{\charEnhancementOne}{\textbf{Kinetic Shield} (1 armor)}
\renewcommand{\charEnhancementTwo}{\textbf{Magnetron Core} (+1 on Ride or Slide)}
\renewcommand{\charEnhancementThree}{\textbf{Warcry} (+1 to \emph{Team-Up!})}
\renewcommand{\charEnhancementFour}{\textbf{Ancient Katana from your Mentor} (2 harm, hand, slash, pierce)}
\renewcommand{\charEnhancementFive}{\textbf{MutaTech Vibro Polearm} (1 harm, close, slash, pierce)}
\renewcommand{\charEnhancementSix}{\textbf{Tactical HUD} (Always know the condition of team mates)}

\renewcommand{\moveOne}{\textbf{Keep tight, hit hard.} When you coordinate the team, roll +Discipline. On a hit you raise the Team Mojo by 1. On a 10+ you additionally get +1 forward.}
\renewcommand{\moveTwo}{\textbf{Semper paratis.} whenever your fall back on your preperations, roll +Style. On a 10+ you have just the thing. On a 7-9 you do not have the perfect solution, but something close, ask the GM what.}
\renewcommand{\moveThree}{\textbf{I see the play.} Whenever you read a tactical situation roll +Discipline. On a hit, ask the GM one question, on a 10+ two.
\vspace{-6pt}
\begin{itemize}
    \setlength\itemsep{-0.5em}
    \item Where is a weak spot?
    \item What's the biggest threat?
    \item Who is out of position?
    \item What is there to do to end this fast?
\end{itemize}
}
\renewcommand{\moveFour}{\textbf{Not on my watch.} Once per scene, you can take the harm or condition of others.}
\renewcommand{\moveFive}{\textbf{I've seen this before.} Once per session, you can declare how you trained for this exact scenario and gain +1 forward.}

\renewcommand{\relationsOne}{\rule{2cm}{1pt} saw you fail. In what situation?}
\renewcommand{\relationsTwo}{You took the fall for \rule{2cm}{1pt}. Ask the player for what.}
\renewcommand{\relationsThree}{\rule{2cm}{1pt} made a mixtape. You hate 90\% of it. You never turn it off. Ask the player what music is on it.}

\renewcommand{\leadingPrinciplesOne}{Hear everyone out}
\renewcommand{\leadingPrinciplesTwo}{Moderate a discussion}
\renewcommand{\leadingPrinciplesThree}{Decide}
\renewcommand{\leadingPrinciplesFour}{Find compromises}

\renewcommand{\flipsideOne}{Feel helpless}
\renewcommand{\flipsideTwo}{Compromises are not always the best for everyone}
\renewcommand{\flipsideThree}{You come too short}
\renewcommand{\flipsideFour}{You chose the wrong course of action}

\pagenumbering{gobble}

\vspace*{\fill}

\begin{figure}[h!]
\centering\includegraphics[width=12cm]{\playbookImage}
\vspace{-\baselineskip}\vspace{+0.1pt}
\rule{\linewidth}{2pt}
\end{figure}
\Huge\fontspec{TradeWinds-Regular.ttf}The \playbookTitle

\normalfont\large
\medskip

\flavorText

\newpage

\Large\fontspec{TradeWinds-Regular.ttf}Animal

\medskip

\normalfont\large \charAnimals

\medskip

\Large\fontspec{TradeWinds-Regular.ttf}Names

\medskip

\normalfont\large \charNames

\medskip

\Large\fontspec{TradeWinds-Regular.ttf}Enhancements

\medskip

\normalfont\large Pick or roll two:

\(\square\) \charEnhancementOne

\(\square\) \charEnhancementTwo

\(\square\) \charEnhancementThree

\(\square\) \charEnhancementFour

\(\square\) \charEnhancementFive

\(\square\) \charEnhancementSix

\medskip

\Large\fontspec{TradeWinds-Regular.ttf}Advancements \(\LARGE \Circle \Circle \Circle \Circle \Circle \)

\medskip

\normalfont\large

\begin{tabular}{l @{\hspace{2cm}} l}
\(\square\) Get +1 Power, max +3 & \(\square\) Take another \playbookTitle~move \\
\(\square\) Get +1 Cool, max +3 & \(\square\) Take another \playbookTitle~enhancement \\
\(\square\) Get +1 Wits, max +3 & \(\square\) Take a move from another playbook \\
\(\square\) Get +1 Heart, max +3 & \(\square\) Take a move from another playbook \\
\(\square\) Get +1 Weird, max +3 & \\
\end{tabular}

\medskip

\Large\fontspec{TradeWinds-Regular.ttf}Conditions

\medskip

\normalfont\large

\(\square\) \textbf{Exposed} (-2 to Power until you eliminate or evade the skeptical)\\
\(\square\) \textbf{Angry} (-2 to Cool until you hurt someone or break something)\\
\(\square\) \textbf{Stressed} (-2 to Weird until you say sth hurtful to someone)\\
\(\square\) \textbf{Jealous} (-2 to Heart until you go on an ego trip)\\
\(\square\) \textbf{Insecure} (-2 to Charm until you take a comment to wrong way)\\
\(\square\) \textbf{Poisoned}  (-1 to all stats until healed)

\medskip

\Large\fontspec{TradeWinds-Regular.ttf}Harm \(\bigtriangledown \bigtriangledown \bigtriangledown \bigtriangledown | \bigtriangledown \bigtriangledown\)

\medskip

\Large\fontspec{TradeWinds-Regular.ttf}Stats

\normalfont\Huge

%Add tabular?
\faBomb~\textcolor{lightgray}{\faCircle[regular]} \hspace{0.5cm} \faStar~\textcolor{lightgray}{\faCircle[regular]} \hspace{0.5cm} \faHeart~\textcolor{lightgray}{\faCircle[regular]} \hspace{0.5cm} \faBrain~\textcolor{lightgray}{\faCircle[regular]} \hspace{0.5cm} \faPizzaSlice~\textcolor{lightgray}{\faCircle[regular]}

\medskip

\Large\fontspec{TradeWinds-Regular.ttf}Notes

\newpage

\Large\fontspec{TradeWinds-Regular.ttf}\playbookTitle~Moves

\medskip

\normalfont\large

\begin{itemize}[label=$\square$]

\item \moveOne

\item \moveTwo

\item \moveThree

\item \moveFour

\item \moveFive

\end{itemize}


\Large\fontspec{TradeWinds-Regular.ttf}Relations

\medskip

\normalfont\large

\begin{itemize}[label=$\square$]
    \item \relationsOne
    \item \relationsTwo
    \item \relationsThree
\end{itemize}

\begin{tabular}{l @{\hspace{2cm}} l}

\Large\fontspec{TradeWinds-Regular.ttf}Leading Principles & \Large\fontspec{TradeWinds-Regular.ttf}Flipside \medskip\\

\normalfont\large

$\bullet$ \leadingPrinciplesOne & $\bullet$ \flipsideOne \\
$\bullet$ \leadingPrinciplesTwo &  $\bullet$ \flipsideTwo \\
$\bullet$ \leadingPrinciplesThree &  $\bullet$ \flipsideThree \\
$\bullet$ \leadingPrinciplesFour &  $\bullet$ \flipsideFour \\

\end{tabular}

\newpage
\renewcommand{\playbookTitle}{Medic}

\renewcommand{\playbookImage}{images/medic.png}

\renewcommand{\flavorText}{
\textit{I won't let you die. You are everything to me.}\\
\medskip
You know a lot about medicine and the treatment of injuries. You are empathic and care about others.\\
\medskip
Others come to you for guidance or just to talk. They like having you around. You give them a feeling of serenity.
}

\renewcommand{\charNames}{1.Aeris, 2.Doc, 3.Caelum, 4.Orin, 5.Kiora, 6.Pulse, \rule{2cm}{1pt}}

\renewcommand{\charAnimals}{1.Dog, 2.Axolotl, 3.Squirrel, 4.Panda, 5.Capybara, 6.Koala, \rule{2cm}{1pt}}

\renewcommand{\charEnhancementOne}{\textbf{Tranquilizer Gun} (1 harm, far, non-lethal)}
\renewcommand{\charEnhancementTwo}{\textbf{Adrenaline Sync Gland} (Once per session, give +1 forward)}
\renewcommand{\charEnhancementThree}{\textbf{Dual-Wield Scalpels} (2 harm, hand, slash)}
\renewcommand{\charEnhancementFour}{\textbf{Physiology of Contemporary Fauna Volume III} (Let's you analyze biological, medical and anatomical details.)}
\renewcommand{\charEnhancementFive}{\textbf{Gaseous Healing Enzyme} (Once per session, you can heal 1 harm from all allies.)}
\renewcommand{\charEnhancementSix}{\textbf{SymbioPack Med-Kit} (Heal 4 harm, one time use)}

%Yin-Yang, Heal, trade harm, drunken master, thousand strikes
\renewcommand{\moveOne}{\textbf{You're not dying on my watch.} When you tend to the wounds of an ally, roll +Wits. On a 7-9 you heal 1 harm, on a 10+ 2 harm.}
\renewcommand{\moveTwo}{\textbf{Here, take this.} Explain what you give your ally to boost them. On a 10+ it gives them +2 forward. On a 7-9 it gives them +1 forward but it has unforeseen side effects. The ally explain which.}
\renewcommand{\moveThree}{\textbf{Nobody is left behind.} You can activate gaseous healing enzymes. When you do so, roll +Grit. On a 10+, every ally heals one harm or condition. On a 7-9, you heal the harm of one ally and chose one:
\vspace{-6pt}
\begin{itemize}
    \setlength\itemsep{-0.5em}
    \item The is affected enemy as well. Explain how.
    \item Unforeseen side effects occur. The ally eplains which.
    \item It drains you. You have to take a breath.
\end{itemize}
}
\renewcommand{\moveFour}{\textbf{Good news: You're fixable.} When you empathically talk to an ally, roll +Discipline. On a 7-9 you heal one condition. On a 10+ up to two conditions.}
\renewcommand{\moveFive}{\textbf{Get up. We need you.} When you take care of a downed ally, roll on +Discipline. On a hit, you stabilize the ally. On a 7-9 it drains you and you have to rest for a minute.}

\renewcommand{\relationsOne}{You healed an insane injury on \rule{2cm}{1pt}. Ask the player what it was and how they got it.}
\renewcommand{\relationsTwo}{\rule{2cm}{1pt}'s extraordinary physiology needs special care you provide. Ask the player what it is.}
\renewcommand{\relationsThree}{Something about \rule{2cm}{1pt}'s lifestyle is incredibly unhealthy. What is it?}

\renewcommand{\leadingPrinciplesOne}{Support others}
\renewcommand{\leadingPrinciplesTwo}{Take yourself back}
\renewcommand{\leadingPrinciplesThree}{Care about others}
\renewcommand{\leadingPrinciplesFour}{Be concerned}

\renewcommand{\flipsideOne}{Be scared}
\renewcommand{\flipsideTwo}{Blame yourself}
\renewcommand{\flipsideThree}{Remind others that something is unhealthy}
\renewcommand{\flipsideFour}{Be overprotective}

\pagenumbering{gobble}

\vspace*{\fill}

\begin{figure}[h!]
\centering\includegraphics[width=12cm]{\playbookImage}
\vspace{-\baselineskip}\vspace{+0.1pt}
\rule{\linewidth}{2pt}
\end{figure}
\Huge\fontspec{TradeWinds-Regular.ttf}The \playbookTitle

\normalfont\large
\medskip

\flavorText

\newpage

\Large\fontspec{TradeWinds-Regular.ttf}Animal

\medskip

\normalfont\large \charAnimals

\medskip

\Large\fontspec{TradeWinds-Regular.ttf}Names

\medskip

\normalfont\large \charNames

\medskip

\Large\fontspec{TradeWinds-Regular.ttf}Enhancements

\medskip

\normalfont\large Pick or roll two:

\(\square\) \charEnhancementOne

\(\square\) \charEnhancementTwo

\(\square\) \charEnhancementThree

\(\square\) \charEnhancementFour

\(\square\) \charEnhancementFive

\(\square\) \charEnhancementSix

\medskip

\Large\fontspec{TradeWinds-Regular.ttf}Advancements \(\LARGE \Circle \Circle \Circle \Circle \Circle \)

\medskip

\normalfont\large

\begin{tabular}{l @{\hspace{2cm}} l}
\(\square\) Get +1 Power, max +3 & \(\square\) Take another \playbookTitle~move \\
\(\square\) Get +1 Cool, max +3 & \(\square\) Take another \playbookTitle~enhancement \\
\(\square\) Get +1 Wits, max +3 & \(\square\) Take a move from another playbook \\
\(\square\) Get +1 Heart, max +3 & \(\square\) Take a move from another playbook \\
\(\square\) Get +1 Weird, max +3 & \\
\end{tabular}

\medskip

\Large\fontspec{TradeWinds-Regular.ttf}Conditions

\medskip

\normalfont\large

\(\square\) \textbf{Exposed} (-2 to Power until you eliminate or evade the skeptical)\\
\(\square\) \textbf{Angry} (-2 to Cool until you hurt someone or break something)\\
\(\square\) \textbf{Stressed} (-2 to Weird until you say sth hurtful to someone)\\
\(\square\) \textbf{Jealous} (-2 to Heart until you go on an ego trip)\\
\(\square\) \textbf{Insecure} (-2 to Charm until you take a comment to wrong way)\\
\(\square\) \textbf{Poisoned}  (-1 to all stats until healed)

\medskip

\Large\fontspec{TradeWinds-Regular.ttf}Harm \(\bigtriangledown \bigtriangledown \bigtriangledown \bigtriangledown | \bigtriangledown \bigtriangledown\)

\medskip

\Large\fontspec{TradeWinds-Regular.ttf}Stats

\normalfont\Huge

%Add tabular?
\faBomb~\textcolor{lightgray}{\faCircle[regular]} \hspace{0.5cm} \faStar~\textcolor{lightgray}{\faCircle[regular]} \hspace{0.5cm} \faHeart~\textcolor{lightgray}{\faCircle[regular]} \hspace{0.5cm} \faBrain~\textcolor{lightgray}{\faCircle[regular]} \hspace{0.5cm} \faPizzaSlice~\textcolor{lightgray}{\faCircle[regular]}

\medskip

\Large\fontspec{TradeWinds-Regular.ttf}Notes

\newpage

\Large\fontspec{TradeWinds-Regular.ttf}\playbookTitle~Moves

\medskip

\normalfont\large

\begin{itemize}[label=$\square$]

\item \moveOne

\item \moveTwo

\item \moveThree

\item \moveFour

\item \moveFive

\end{itemize}


\Large\fontspec{TradeWinds-Regular.ttf}Relations

\medskip

\normalfont\large

\begin{itemize}[label=$\square$]
    \item \relationsOne
    \item \relationsTwo
    \item \relationsThree
\end{itemize}

\begin{tabular}{l @{\hspace{2cm}} l}

\Large\fontspec{TradeWinds-Regular.ttf}Leading Principles & \Large\fontspec{TradeWinds-Regular.ttf}Flipside \medskip\\

\normalfont\large

$\bullet$ \leadingPrinciplesOne & $\bullet$ \flipsideOne \\
$\bullet$ \leadingPrinciplesTwo &  $\bullet$ \flipsideTwo \\
$\bullet$ \leadingPrinciplesThree &  $\bullet$ \flipsideThree \\
$\bullet$ \leadingPrinciplesFour &  $\bullet$ \flipsideFour \\

\end{tabular}

\newpage
\renewcommand{\playbookTitle}{Rebel}

\renewcommand{\playbookImage}{images/rebel.png}

\renewcommand{\flavorText}{
\textit{"I never been in the system, so why obey it?"}\\
\medskip
You have a problem to accept authority. Society as it is, doesn't accept you. So why not change it? Why not overthrow it and shape something new?\\
\medskip
The others might see you as a trouble maker.}

\renewcommand{\charNames}{1.Spike, 2.Fuze, 3.Cherry, 4.Nyx, 5.Slash, 6.Skara, \rule{2cm}{1pt}}

\renewcommand{\charAnimals}{1.Hedgehog, 2.Goat, 3.Wolf, 4.Skunk, 5.Wombat, 6.Frog, \rule{2cm}{1pt}}

\renewcommand{\charEnhancementOne}{\textbf{Baseball Bat with Scripture} (2 harm, hand, blunt, pierce)}
\renewcommand{\charEnhancementTwo}{\textbf{Sledgehammer} (2 harm, hand, blunt, slow)}
\renewcommand{\charEnhancementThree}{\textbf{Can of Gasoline} (Wann see something burn? One time use.)}
\renewcommand{\charEnhancementFour}{\textbf{Acid Spray Gland} (Ignores 1 armor)}
\renewcommand{\charEnhancementFive}{\textbf{Homemade Proximity Mines} (2 harm, loud.)}
\renewcommand{\charEnhancementSix}{\textbf{Denim Vest with Spikes and Patches} (1 armor, +1 Style)}

\renewcommand{\moveOne}{\textbf{We are the rust upon your gears.} When you give a rebellious song, speech or action to others, roll on \emph{Talk the Talk} +Grit.}
\renewcommand{\moveThree}{\textbf{Bring down the big man!} When you do something to annoy authority or big companies +1 forward.}
\renewcommand{\moveFour}{\textbf{Yeah, I build these explosives myself.... so what?} When you rig something with explosives, roll +Grit. On a 10+ it works just as expected. On a 7-9 it works, but something goes wrong and every ally either gets 1 harm or one condition (their choice).}
\renewcommand{\moveTwo}{\textbf{We are legion.} You are part of an anti-authoritarian hacker collective. When you call in a favor, roll +Wits. On a 10+, choose one of the following
\vspace{-6pt}
\begin{itemize}
    \setlength\itemsep{-0.5em}
    \item Something public or corporate shuts down
    \item Get some (delicate) piece of information
    \item Change a numbers on a public or corporate digital display
\end{itemize}}\renewcommand{\moveFive}{\textbf{Shh… the pigs are squealing.} A modified walkman allows you to eavesdrop on authority frequencies. When you try to find out what they are up to, roll +Wits. On a 10+ you get a clear idea of the situation. On a 7-9 the connection is noisy and you can only hear fragments.}

\renewcommand{\relationsOne}{\rule{2cm}{1pt} and you hate the same insitution or company. What is it? What do they do?}
\renewcommand{\relationsTwo}{You and \rule{2cm}{1pt} stole something from an authority or company. What is it?}
\renewcommand{\relationsThree}{\rule{2cm}{1pt} and you got into trouble with the police. How come? What was it about?}

\renewcommand{\leadingPrinciplesOne}{Change the direction}
\renewcommand{\leadingPrinciplesTwo}{Activate others}
\renewcommand{\leadingPrinciplesThree}{Give speeches}
\renewcommand{\leadingPrinciplesFour}{Be connected}

\renewcommand{\flipsideOne}{Be disruptive}
\renewcommand{\flipsideTwo}{Go against authority}
\renewcommand{\flipsideThree}{Don't let others dictate what you do}
\renewcommand{\flipsideFour}{Break free}

\pagenumbering{gobble}

\vspace*{\fill}

\begin{figure}[h!]
\centering\includegraphics[width=12cm]{\playbookImage}
\vspace{-\baselineskip}\vspace{+0.1pt}
\rule{\linewidth}{2pt}
\end{figure}
\Huge\fontspec{TradeWinds-Regular.ttf}The \playbookTitle

\normalfont\large
\medskip

\flavorText

\newpage

\Large\fontspec{TradeWinds-Regular.ttf}Animal

\medskip

\normalfont\large \charAnimals

\medskip

\Large\fontspec{TradeWinds-Regular.ttf}Names

\medskip

\normalfont\large \charNames

\medskip

\Large\fontspec{TradeWinds-Regular.ttf}Enhancements

\medskip

\normalfont\large Pick or roll two:

\(\square\) \charEnhancementOne

\(\square\) \charEnhancementTwo

\(\square\) \charEnhancementThree

\(\square\) \charEnhancementFour

\(\square\) \charEnhancementFive

\(\square\) \charEnhancementSix

\medskip

\Large\fontspec{TradeWinds-Regular.ttf}Advancements \(\LARGE \Circle \Circle \Circle \Circle \Circle \)

\medskip

\normalfont\large

\begin{tabular}{l @{\hspace{2cm}} l}
\(\square\) Get +1 Power, max +3 & \(\square\) Take another \playbookTitle~move \\
\(\square\) Get +1 Cool, max +3 & \(\square\) Take another \playbookTitle~enhancement \\
\(\square\) Get +1 Wits, max +3 & \(\square\) Take a move from another playbook \\
\(\square\) Get +1 Heart, max +3 & \(\square\) Take a move from another playbook \\
\(\square\) Get +1 Weird, max +3 & \\
\end{tabular}

\medskip

\Large\fontspec{TradeWinds-Regular.ttf}Conditions

\medskip

\normalfont\large

\(\square\) \textbf{Exposed} (-2 to Power until you eliminate or evade the skeptical)\\
\(\square\) \textbf{Angry} (-2 to Cool until you hurt someone or break something)\\
\(\square\) \textbf{Stressed} (-2 to Weird until you say sth hurtful to someone)\\
\(\square\) \textbf{Jealous} (-2 to Heart until you go on an ego trip)\\
\(\square\) \textbf{Insecure} (-2 to Charm until you take a comment to wrong way)\\
\(\square\) \textbf{Poisoned}  (-1 to all stats until healed)

\medskip

\Large\fontspec{TradeWinds-Regular.ttf}Harm \(\bigtriangledown \bigtriangledown \bigtriangledown \bigtriangledown | \bigtriangledown \bigtriangledown\)

\medskip

\Large\fontspec{TradeWinds-Regular.ttf}Stats

\normalfont\Huge

%Add tabular?
\faBomb~\textcolor{lightgray}{\faCircle[regular]} \hspace{0.5cm} \faStar~\textcolor{lightgray}{\faCircle[regular]} \hspace{0.5cm} \faHeart~\textcolor{lightgray}{\faCircle[regular]} \hspace{0.5cm} \faBrain~\textcolor{lightgray}{\faCircle[regular]} \hspace{0.5cm} \faPizzaSlice~\textcolor{lightgray}{\faCircle[regular]}

\medskip

\Large\fontspec{TradeWinds-Regular.ttf}Notes

\newpage

\Large\fontspec{TradeWinds-Regular.ttf}\playbookTitle~Moves

\medskip

\normalfont\large

\begin{itemize}[label=$\square$]

\item \moveOne

\item \moveTwo

\item \moveThree

\item \moveFour

\item \moveFive

\end{itemize}


\Large\fontspec{TradeWinds-Regular.ttf}Relations

\medskip

\normalfont\large

\begin{itemize}[label=$\square$]
    \item \relationsOne
    \item \relationsTwo
    \item \relationsThree
\end{itemize}

\begin{tabular}{l @{\hspace{2cm}} l}

\Large\fontspec{TradeWinds-Regular.ttf}Leading Principles & \Large\fontspec{TradeWinds-Regular.ttf}Flipside \medskip\\

\normalfont\large

$\bullet$ \leadingPrinciplesOne & $\bullet$ \flipsideOne \\
$\bullet$ \leadingPrinciplesTwo &  $\bullet$ \flipsideTwo \\
$\bullet$ \leadingPrinciplesThree &  $\bullet$ \flipsideThree \\
$\bullet$ \leadingPrinciplesFour &  $\bullet$ \flipsideFour \\

\end{tabular}

\newpage
\renewcommand{\playbookTitle}{Rogue}
%Scoundrel? A little more ressources?

\renewcommand{\playbookImage}{images/rogue.png}

\renewcommand{\flavorText}{
\textit{"I move silently in and out. Nobody will know that I've been there."}\\
\medskip
Your mutations made you the perfect hunter. And thief, for that matter. You move silently and strike with precision. None of your moves is brute or random.\\
\medskip
Others react to you at times skeptical. But they also value your abilities. You are a valuable addition to the team and they know it.}

\renewcommand{\charNames}{1.Whisper, 2.Zane, 3.Blink, 4.Mox, 5.Lux, 6.Vale, \rule{2cm}{1pt}}

\renewcommand{\charAnimals}{1.Snake, 2.Cat, 3.Rat, 4.Chameleon, 5.Bat, 6.Raccoon, \rule{2cm}{1pt}}

\renewcommand{\charEnhancementOne}{\textbf{Mimicking Vocal Cords} (You can mimic sounds and voices you heard.)}
\renewcommand{\charEnhancementTwo}{\textbf{Retractable Subskin Blades} (2 harm, 4 harm when unseen, hand, pierce, hidden)}
\renewcommand{\charEnhancementThree}{\textbf{Shape-Morphing Skinstructure} (+1 on \emph{Blend In})}
\renewcommand{\charEnhancementFour}{\textbf{Venom} (Your weapon ignores 1 armor)}
\renewcommand{\charEnhancementFive}{\textbf{Gecko Palms} (Climb most surfaces)}
\renewcommand{\charEnhancementSix}{\textbf{Electro-Magnetic Neural Lobe} (Jam frequencies)}
%Add more weapons?

\renewcommand{\moveOne}{\textbf{I don't do goodbyes.} When you try to vanish into thin air, explain how you do it and roll +Style. On a 10+ you vanish without a trace. On a 7-9 you leave something behind.}
\renewcommand{\moveTwo}{\textbf{Become one with the shadows.} You can use \emph{Blend In} to blend into surroundings and shadows.}
\renewcommand{\moveThree}{\textbf{If they didn't want it stolen, they should've hidden it better.} When you attempt to steal something roll +Style. On a 10+ you steal it without a trace. On a 7-9 you leave something behind.}
\renewcommand{\moveFour}{\textbf{Nice face. I hope it's okay if I borrow it.} Your mutation allows for you to shapeshift into someone you seen in person for a short time. Roll +Discipline. On a 10+ you successfully shapeshift. On a 7-9 you shapeshift, but choose one of the following:
\vspace{-6pt}
\begin{itemize}
    \setlength\itemsep{-0.5em}
    \item The duration is shorter than you expected
    \item One noticable detail about the person is off. Explain what.
    \item You do something so out of character for the person. Explain what.
\end{itemize}}
\renewcommand{\moveFive}{\textbf{No, I don't have a hairpin, but how about lockpicks?} When you try to pick a lock, roll +Discipline. On a hit you open the door. On a 7-9 additionally choose one:
\vspace{-6pt}
\begin{itemize}
    \setlength\itemsep{-0.5em}
    \item You draw attention
    \item Behind the door was not what you expected
    \item Somebody is right behind the door
    \item Your tools break. You can't use this move until next session
\end{itemize}}

\renewcommand{\relationsOne}{\rule{2cm}{1pt} and you stole something together. What was it?}
\renewcommand{\relationsTwo}{You move perfectly silent out of habit and walked in on \rule{2cm}{1pt}. Ask the player what embarrassing situation they were in.}
\renewcommand{\relationsThree}{You overheard a conversation of \rule{2cm}{1pt} and \rule{2cm}{1pt}. Ask the players what it was about.}

\renewcommand{\leadingPrinciplesOne}{Be discrete}
\renewcommand{\leadingPrinciplesTwo}{Move silently}
\renewcommand{\leadingPrinciplesThree}{Take hidden routes}
\renewcommand{\leadingPrinciplesFour}{Reveal and trade secrets}

\renewcommand{\flipsideOne}{Take everything for yourself}
\renewcommand{\flipsideTwo}{Steal}
\renewcommand{\flipsideThree}{Reveal something bad}
\renewcommand{\flipsideFour}{Think about yourself}

\pagenumbering{gobble}

\vspace*{\fill}

\begin{figure}[h!]
\centering\includegraphics[width=12cm]{\playbookImage}
\vspace{-\baselineskip}\vspace{+0.1pt}
\rule{\linewidth}{2pt}
\end{figure}
\Huge\fontspec{TradeWinds-Regular.ttf}The \playbookTitle

\normalfont\large
\medskip

\flavorText

\newpage

\Large\fontspec{TradeWinds-Regular.ttf}Animal

\medskip

\normalfont\large \charAnimals

\medskip

\Large\fontspec{TradeWinds-Regular.ttf}Names

\medskip

\normalfont\large \charNames

\medskip

\Large\fontspec{TradeWinds-Regular.ttf}Enhancements

\medskip

\normalfont\large Pick or roll two:

\(\square\) \charEnhancementOne

\(\square\) \charEnhancementTwo

\(\square\) \charEnhancementThree

\(\square\) \charEnhancementFour

\(\square\) \charEnhancementFive

\(\square\) \charEnhancementSix

\medskip

\Large\fontspec{TradeWinds-Regular.ttf}Advancements \(\LARGE \Circle \Circle \Circle \Circle \Circle \)

\medskip

\normalfont\large

\begin{tabular}{l @{\hspace{2cm}} l}
\(\square\) Get +1 Power, max +3 & \(\square\) Take another \playbookTitle~move \\
\(\square\) Get +1 Cool, max +3 & \(\square\) Take another \playbookTitle~enhancement \\
\(\square\) Get +1 Wits, max +3 & \(\square\) Take a move from another playbook \\
\(\square\) Get +1 Heart, max +3 & \(\square\) Take a move from another playbook \\
\(\square\) Get +1 Weird, max +3 & \\
\end{tabular}

\medskip

\Large\fontspec{TradeWinds-Regular.ttf}Conditions

\medskip

\normalfont\large

\(\square\) \textbf{Exposed} (-2 to Power until you eliminate or evade the skeptical)\\
\(\square\) \textbf{Angry} (-2 to Cool until you hurt someone or break something)\\
\(\square\) \textbf{Stressed} (-2 to Weird until you say sth hurtful to someone)\\
\(\square\) \textbf{Jealous} (-2 to Heart until you go on an ego trip)\\
\(\square\) \textbf{Insecure} (-2 to Charm until you take a comment to wrong way)\\
\(\square\) \textbf{Poisoned}  (-1 to all stats until healed)

\medskip

\Large\fontspec{TradeWinds-Regular.ttf}Harm \(\bigtriangledown \bigtriangledown \bigtriangledown \bigtriangledown | \bigtriangledown \bigtriangledown\)

\medskip

\Large\fontspec{TradeWinds-Regular.ttf}Stats

\normalfont\Huge

%Add tabular?
\faBomb~\textcolor{lightgray}{\faCircle[regular]} \hspace{0.5cm} \faStar~\textcolor{lightgray}{\faCircle[regular]} \hspace{0.5cm} \faHeart~\textcolor{lightgray}{\faCircle[regular]} \hspace{0.5cm} \faBrain~\textcolor{lightgray}{\faCircle[regular]} \hspace{0.5cm} \faPizzaSlice~\textcolor{lightgray}{\faCircle[regular]}

\medskip

\Large\fontspec{TradeWinds-Regular.ttf}Notes

\newpage

\Large\fontspec{TradeWinds-Regular.ttf}\playbookTitle~Moves

\medskip

\normalfont\large

\begin{itemize}[label=$\square$]

\item \moveOne

\item \moveTwo

\item \moveThree

\item \moveFour

\item \moveFive

\end{itemize}


\Large\fontspec{TradeWinds-Regular.ttf}Relations

\medskip

\normalfont\large

\begin{itemize}[label=$\square$]
    \item \relationsOne
    \item \relationsTwo
    \item \relationsThree
\end{itemize}

\begin{tabular}{l @{\hspace{2cm}} l}

\Large\fontspec{TradeWinds-Regular.ttf}Leading Principles & \Large\fontspec{TradeWinds-Regular.ttf}Flipside \medskip\\

\normalfont\large

$\bullet$ \leadingPrinciplesOne & $\bullet$ \flipsideOne \\
$\bullet$ \leadingPrinciplesTwo &  $\bullet$ \flipsideTwo \\
$\bullet$ \leadingPrinciplesThree &  $\bullet$ \flipsideThree \\
$\bullet$ \leadingPrinciplesFour &  $\bullet$ \flipsideFour \\

\end{tabular}

\newpage
\input{playbooks/seducer.tex}
\input{playbooks/shaolin.tex}
\renewcommand{\playbookTitle}{Wildcard}

\renewcommand{\playbookImage}{images/wildcard.png}

\renewcommand{\flavorText}{
Fast-talking, fast-riding, always radical. Every deck needs a wild card.\\
\medskip
You are chaotic, stylish and emotional. If you don't feel it, it's not happening. You do what you love and you do it with style.
}

\renewcommand{\charNames}{1.Sparx, 2.Jinx, 3.Blitz, 4.Fizz, 5.Zeke, 6.Turbo, \rule{2cm}{1pt}}

\renewcommand{\charAnimals}{1.Platypus, 2.Sloth, 3.Otter, 4.Gecko, 5.Red Panda, 6.Anteater, \rule{2cm}{1pt}}

\renewcommand{\charEnhancementOne}{\textbf{Molotow Graffiti Cans} (1 harm, area, close)}
\renewcommand{\charEnhancementTwo}{\textbf{Skateboard} (+1 on Ride or Slide)}
\renewcommand{\charEnhancementThree}{\textbf{Razorwire BladeYoyo} (2 harm, close)}
\renewcommand{\charEnhancementFour}{\textbf{Lucky coin} (once per session, reroll)}
\renewcommand{\charEnhancementFive}{\textbf{Hyperflex Skeleton} (no falling damage)}
\renewcommand{\charEnhancementSix}{\textbf{Sprayer Mask} (immune to gaseous toxins)}

\renewcommand{\moveOne}{\textbf{I'm just a veeery weird cosplayer.} Get +1 when trying to Blend In}
\renewcommand{\moveTwo}{
\textbf{10\% skill, 90\% bad decisions!} Whenever you do some form of stunt, roll +Cool. On a hit choose one of the following
\begin{itemize}
    \item Get somewhere no-one else can
    \item Get +1 forward
    \item Impress or distract
    \item Create a new route or shortcut
\end{itemize}
On a 7-9 additionally choose one of the following
\begin{itemize}
    \item Get 1 harm
    \item Get a condition
    \item Draw unwanted attention
    \item You're off-balance or vulnerable for a moment
\end{itemize}
}
\renewcommand{\moveThree}{\textbf{Art school's overrated.} You can get information from graffitis. You can tag to influence other sprayers. Orientation in the city is therefore enhanced.}
\renewcommand{\moveFour}{\textbf{Style points? Maxed out.} You weave style and tricks into your fighting and become unpredictable. When you Kick Shell! you can roll +Weird.}
\renewcommand{\moveFive}{\textbf{I licked it already. You're welcome.} You always have some form of junk food with you. When you offer it to someone, they roll +Weird. One a hit, they remove 1 harm, on a 7-9 they get a condition.}

\renewcommand{\relationsOne}{You and \rule{2cm}{1pt} obsess over the same junk food. What is it?}
\renewcommand{\relationsTwo}{\rule{2cm}{1pt} found your art and was surprised by it. What did you draw?}
\renewcommand{\relationsThree}{You taught \rule{2cm}{1pt} some tricks. In what sport and what tricks?}

\renewcommand{\leadingPrinciplesOne}{Be chaotic}
\renewcommand{\leadingPrinciplesTwo}{Think outside the box}
\renewcommand{\leadingPrinciplesThree}{Live in the moment}
\renewcommand{\leadingPrinciplesFour}{Have passion}

\renewcommand{\flipsideOne}{Seek crazy experiences}
\renewcommand{\flipsideTwo}{Get addicted to something}
\renewcommand{\flipsideThree}{Avoid being bored at all cost}
\renewcommand{\flipsideFour}{Put yourself or others in danger}

\pagenumbering{gobble}

\vspace*{\fill}

\begin{figure}[h!]
\centering\includegraphics[width=12cm]{\playbookImage}
\vspace{-\baselineskip}\vspace{+0.1pt}
\rule{\linewidth}{2pt}
\end{figure}
\Huge\fontspec{TradeWinds-Regular.ttf}The \playbookTitle

\normalfont\large
\medskip

\flavorText

\newpage

\Large\fontspec{TradeWinds-Regular.ttf}Animal

\medskip

\normalfont\large \charAnimals

\medskip

\Large\fontspec{TradeWinds-Regular.ttf}Names

\medskip

\normalfont\large \charNames

\medskip

\Large\fontspec{TradeWinds-Regular.ttf}Enhancements

\medskip

\normalfont\large Pick or roll two:

\(\square\) \charEnhancementOne

\(\square\) \charEnhancementTwo

\(\square\) \charEnhancementThree

\(\square\) \charEnhancementFour

\(\square\) \charEnhancementFive

\(\square\) \charEnhancementSix

\medskip

\Large\fontspec{TradeWinds-Regular.ttf}Advancements \(\LARGE \Circle \Circle \Circle \Circle \Circle \)

\medskip

\normalfont\large

\begin{tabular}{l @{\hspace{2cm}} l}
\(\square\) Get +1 Power, max +3 & \(\square\) Take another \playbookTitle~move \\
\(\square\) Get +1 Cool, max +3 & \(\square\) Take another \playbookTitle~enhancement \\
\(\square\) Get +1 Wits, max +3 & \(\square\) Take a move from another playbook \\
\(\square\) Get +1 Heart, max +3 & \(\square\) Take a move from another playbook \\
\(\square\) Get +1 Weird, max +3 & \\
\end{tabular}

\medskip

\Large\fontspec{TradeWinds-Regular.ttf}Conditions

\medskip

\normalfont\large

\(\square\) \textbf{Exposed} (-2 to Power until you eliminate or evade the skeptical)\\
\(\square\) \textbf{Angry} (-2 to Cool until you hurt someone or break something)\\
\(\square\) \textbf{Stressed} (-2 to Weird until you say sth hurtful to someone)\\
\(\square\) \textbf{Jealous} (-2 to Heart until you go on an ego trip)\\
\(\square\) \textbf{Insecure} (-2 to Charm until you take a comment to wrong way)\\
\(\square\) \textbf{Poisoned}  (-1 to all stats until healed)

\medskip

\Large\fontspec{TradeWinds-Regular.ttf}Harm \(\bigtriangledown \bigtriangledown \bigtriangledown \bigtriangledown | \bigtriangledown \bigtriangledown\)

\medskip

\Large\fontspec{TradeWinds-Regular.ttf}Stats

\normalfont\Huge

%Add tabular?
\faBomb~\textcolor{lightgray}{\faCircle[regular]} \hspace{0.5cm} \faStar~\textcolor{lightgray}{\faCircle[regular]} \hspace{0.5cm} \faHeart~\textcolor{lightgray}{\faCircle[regular]} \hspace{0.5cm} \faBrain~\textcolor{lightgray}{\faCircle[regular]} \hspace{0.5cm} \faPizzaSlice~\textcolor{lightgray}{\faCircle[regular]}

\medskip

\Large\fontspec{TradeWinds-Regular.ttf}Notes

\newpage

\Large\fontspec{TradeWinds-Regular.ttf}\playbookTitle~Moves

\medskip

\normalfont\large

\begin{itemize}[label=$\square$]

\item \moveOne

\item \moveTwo

\item \moveThree

\item \moveFour

\item \moveFive

\end{itemize}


\Large\fontspec{TradeWinds-Regular.ttf}Relations

\medskip

\normalfont\large

\begin{itemize}[label=$\square$]
    \item \relationsOne
    \item \relationsTwo
    \item \relationsThree
\end{itemize}

\begin{tabular}{l @{\hspace{2cm}} l}

\Large\fontspec{TradeWinds-Regular.ttf}Leading Principles & \Large\fontspec{TradeWinds-Regular.ttf}Flipside \medskip\\

\normalfont\large

$\bullet$ \leadingPrinciplesOne & $\bullet$ \flipsideOne \\
$\bullet$ \leadingPrinciplesTwo &  $\bullet$ \flipsideTwo \\
$\bullet$ \leadingPrinciplesThree &  $\bullet$ \flipsideThree \\
$\bullet$ \leadingPrinciplesFour &  $\bullet$ \flipsideFour \\

\end{tabular}

\newpage
%\end{comment}
\pagenumbering{gobble}

\vspace*{\fill}

\begin{figure}[h!]
\centering\includegraphics[height=13cm]{images/van}
\vspace{-\baselineskip}\vspace{+0.1pt}
\rule{\linewidth}{2pt}
\end{figure}
\Huge\fontspec{TradeWinds-Regular.ttf}The Mutant Team

\normalfont\large
\medskip

You are a chaotic team of mutants. Taught by a mentor you slowly become ready to stand firm for your city. Who will you be? How do you organize? How do you get around? Who taught you? This is the place to find out.

\newpage

\Large\fontspec{TradeWinds-Regular.ttf}Your City
\normalfont\large
\medskip

What is special about your city? What makes it stand out compared to other cities? What makes it worth to protect?\\

\rule{\linewidth}{1pt}

\medskip

\Large\fontspec{TradeWinds-Regular.ttf}Means of Transport
\normalfont\large


\begin{enumerate}
    \setlength\itemsep{-0.5em}
    \item Van tailored to your needs
    \item Bikes
    \item Hoverboards
    \item Underground tunnels (sewers, subway, …)
    \item Parkouring the rooftops
    \item Hijacked subway car
\end{enumerate}


\Large\fontspec{TradeWinds-Regular.ttf}Origin of Mutation


\normalfont\large 
\begin{enumerate}
    \setlength\itemsep{-0.5em}
    \item Ooze mutating animals
    \item Lab experiment gone wrong
    \item Aliens
    \item New evolution of human life
    \item Ancient gods already
    \item Mutating teenagers
\end{enumerate}


\Large\fontspec{TradeWinds-Regular.ttf}Mentor

\medskip

\normalfont\large 
\begin{enumerate}
    \setlength\itemsep{-0.5em}
    \item A more experienced mutant
    \item Decommissioned AI computer
    \item Janitor
    \item Retired cop
    \item Wannabe headhunter
    \item Scientist
\end{enumerate}


\Large\fontspec{TradeWinds-Regular.ttf}Origin of Your Mentor's Endless Wisdom:


\normalfont\large
\begin{enumerate}
    \setlength\itemsep{-0.5em}
    \item TV
    \item Reddit
    \item Another generation of mentor
    \item Magazines
    \item Alien mind probe
    \item Occult rituals
\end{enumerate}

\Large\fontspec{TradeWinds-Regular.ttf}Hideout / Dojo

\normalfont\large
\begin{enumerate}
    \setlength\itemsep{-0.5em}
    \item Underground (1.Sewer crypt, 2.Closed subway station, 3.Service tunnels)
    \item Junkyard (1.Neon signs, 2.Classic cars, 3.Foreign tech)
    \item Run-down factory (1.Favourite junk food, 2.Spray cans, 3.Toys)
    \item Crashed alien ship (1.Ancient, 2.Hi-tech, 3.Custom modded)
    \item Abandoned rooftop (1.Greenhouse, 2.Graffiti court, 3.Pigeon loft)
    \item Forsaken amusement park (1.Waterfront, 2.Bad part of town, 3.Skate park)
\end{enumerate}

\Large\fontspec{TradeWinds-Regular.ttf}Favourite Junk Food

\normalfont\large

\begin{enumerate}
    \setlength\itemsep{-0.5em}
    \item Pineapple Jalapeno Pizza
    \item Brain Freeze Soda Pops
    \item Chaotic Sludge Tacos
    \item Deep Fried Shrimp Burger
    \item Vulcano Chili Dogs
    \item Oozing Mango Sundaes
\end{enumerate}

\Large\fontspec{TradeWinds-Regular.ttf}You are Trained in the Art of

\normalfont\large

\begin{enumerate}
    \setlength\itemsep{-0.5em}
    \item Ninjutsu
    \item Karate
    \item Kung Fu
    \item Fist to Face
    \item Mixed Martial Arts
    \item Capoeira
\end{enumerate}
\pagenumbering{gobble}

\normalfont\large
\medskip

\Large\fontspec{TradeWinds-Regular.ttf}The Evil
\normalfont\large
\medskip

What is special about your city? What makes it stand out compared to other cities? What makes it worth to protect? \rule{0.5\linewidth}{1pt}

\vspace{0.5cm}


\begin{tabular}{l @{\hspace{2cm}} l}
\Large\fontspec{TradeWinds-Regular.ttf}Type & \Large\fontspec{TradeWinds-Regular.ttf}They want to \\
\normalfont\large 1. Ninjas & \normalfont\large 1. Find and kill you \\
\normalfont\large 2. Aliens & \normalfont\large 2. Capture and experiment on you \\
\normalfont\large 3. Other mutants & \normalfont\large 3. Achieve (more) power\\
\normalfont\large 4. Scientists & \normalfont\large 4. Destroy the city\\
\normalfont\large 5. Regular human assholes & \normalfont\large 5. Get to your mentor\\
\normalfont\large 6. Occult forces & \normalfont\large 6. Prove their genius\medskip\\
~~~\rule{0.3\linewidth}{1pt} & ~~~\rule{0.3\linewidth}{1pt} \medskip\\
\Large\fontspec{TradeWinds-Regular.ttf}They are a & \\
\normalfont\large 1. Clan & \normalfont\large \\ 
\normalfont\large 2. Cult & \normalfont\large \\ 
\normalfont\large 3. Enterprise & \normalfont\large \\ 
\normalfont\large 4. Family & \normalfont\large \\ 
\normalfont\large 5. Cartel & \normalfont\large \\ 
\normalfont\large 6. Secret society & \normalfont\large \medskip\\
~~~\rule{0.3\linewidth}{1pt} & \medskip\\
\end{tabular}

\vspace{0.5cm}

\Large\fontspec{TradeWinds-Regular.ttf}~Example Enemies\medskip\\
\normalfont\large
\begin{tabular}{|l|l|l|l|}
    \hline
    \textbf{Enemy Type} & \textbf{Harm} & \textbf{Damage} & \textbf{Notes} \\
    \hline
    Grunt / Goon / Drone & 1--2 & 1 & Go down in a hit or two. Threat in numbers. \\
    \hline
    Thug / Mutant & 3 & 1--2 & More durable. May have strength or mutations. \\
    \hline
    Elite / Enforcer & 4 & 2 & Skilled fighters. Can briefly keep up with PCs. \\
    \hline
    Mini-Boss / Lieutenant & 5 & 2--3 & Dangerous. May have a unique move or mutation. \\
    \hline
    Boss / Major Villain & 6 & 3 & Serious threat. Might have armor or minions. \\
    \hline
    Omega-Level Threat & 7+ & 3--4 & Endgame danger. Requires teamwork or clever tactics. \\
    \hline
\end{tabular}

\vfill

\begin{tikzpicture}[remember picture, overlay]
  \node[anchor=south] at (current page.south) {
    \includegraphics[height=6cm]{images/samurai.png}
  };
\end{tikzpicture}


\chapter*{\fontspec{TradeWinds-Regular.ttf}Appendix N}
\begin{itemize}
    \item \textbf{Teenage Mutant Ninja Turtles}
    \item \textbf{Street Sharks}
    \item \textbf{Extreme Dinosaurs}
    \item \textbf{Biker Mice from Mars}
\end{itemize}

\end{document}